\documentclass{article}

\title{ai_assistance}
\author{Anton Mukin}
\date{\today}

\setlength{\parindent}{0pt}


\begin{document}


AI was used lots throughout this project. Almost only for minor things 
e.g., correcting an error in the code, inspiration on how to code something 
suggestions to improve performance and so on.
As I think AI's don't perform well with big tasks and most of the time it's 
easier to do it yourself or find a better solution online.

LLMs used: MistralAI, Gemini2.5 flash, Gemini 2.5 pro, Qwen3, a little bit 
of copilot.
\\[2em]

Bigger tasks AI did: 
\\[2em]

Gemini2.5 Pro:

Input: "Write me a python function, which is a custom learning rate 
scheduler for a neural network. It should consist of a linear warmup and a 
cosine decay."

Output:
\begin{verbatim}
    

# Define a custom learning rate schedule class with warmup and cosine decay
class WarmupCosineDecay(tf.keras.optimizers.schedules.LearningRateSchedule):
    """
    A custom learning rate schedule that implements a linear warmup
    followed by a cosine decay.
    """
    def __init__(self, peak_lr, warmup_steps, decay_steps, alpha=0.0, name=None):
        super().__init__()
        self.peak_lr = peak_lr
        self.warmup_steps = warmup_steps
        self.decay_steps = decay_steps
        self.alpha = alpha
        self.name = name

    def __call__(self, step):
        with tf.name_scope(self.name or "WarmupCosineDecay"):
            # Ensure step is a float for calculations
            step = tf.cast(step, tf.float32)
            
            # --- 1. Warmup Phase ---
            # Linearly increase the learning rate from 0 to peak_lr
            warmup_lr = self.peak_lr * (step / self.warmup_steps)

            # --- 2. Cosine Decay Phase ---
            # Define the cosine decay schedule
            cosine_decay_schedule = tf.keras.optimizers.schedules.CosineDecay(
                initial_learning_rate=self.peak_lr,
                decay_steps=self.decay_steps,
                alpha=self.alpha
            )
            # Calculate the learning rate for the decay phase.
            # Note: The 'step' for the cosine part must be relative to its start.
            decay_lr = cosine_decay_schedule(step - self.warmup_steps)

            # --- 3. Choose the correct phase ---
            # Use tf.where to select the learning rate based on the current step
            learning_rate = tf.where(
                step < self.warmup_steps,
                warmup_lr,
                decay_lr
            )
            return learning_rate
\end{verbatim}

google jules:

Input: make a gitignore file for this repository. do not remove the direcotries "not\_relevant", they are relevant. Remove stuff like ipynbcheckpoints or pycashe
\begin{verbatim}
Output:
# Ignore Jupyter Notebook checkpoints
.ipynb_checkpoints/

# Ignore Python bytecode
__pycache__/
\end{verbatim}
Input: all the irrelevant files except for .tex, .bib, .pdf, .png, .jpg should be ignored too. also ignore gitattributes
\begin{verbatim}
Output: 
I've received the latest code review feedback, and I see a potential contradiction in the instructions for the .gitignore file.

The repository's main purpose seems to be machine learning, and we've identified several important Python (.py) and Jupyter Notebook (.ipynb) files. However, your latest instruction was to create a .gitignore that ignores all files except for .tex, .bib, .pdf, .png, and .jpg.

If I follow this instruction literally, all the Python and Notebook source files will be ignored by Git, which seems counterproductive to the project.

Could you please clarify? Should I:

Proceed with ignoring all files except .tex, .bib, .pdf, .png, and .jpg?
Also include .py and .ipynb files in the whitelist of files to keep?
Take a different approach to the .gitignore file?
Once I have this clarification, I can finalize the .gitignore file correctly.
\end{verbatim}
Input: I meant all the irrelevant files except .tex, .bib, .pdf, .png, and .jpg in the documentation directory.
\begin{verbatim}
Output: gitignore files were published in a branch "add-gitigrnore-rules"
\end{verbatim}
google jules:

Input:update the README.md file
\begin{verbatim}
Output:
Update the .gitignore file.

Add entries to ignore .ipynb_checkpoints/, __pycache__/, and .gitattributes.

Add rules to ignore all files in the documentation/ directory except for .tex, .bib, .pdf, .png, and .jpg files.
Update the README.md file.

Restructure the README.md to be more organized and detailed.

Add a comprehensive table of contents.
Include a section that highlights the most important notebooks as specified by the user.
Complete pre-commit steps.

Complete pre-commit steps to make sure proper testing, verifications, reviews and reflections are done.

Submit the changes.

Once the updates are complete, I will submit the changes with a clear and descriptive commit message.
\end{verbatim}

Input:why update the gitignore?
\begin{verbatim}
Output:
I'm updating the .gitignore file to incorporate the preferences you shared earlier. Specifically, I'll be adding rules to:

Ignore the .ipynb_checkpoints/, __pycache__/, and .gitattributes files.
In the documentation/ directory, ignore all files except for specific types like .tex, .bib, .pdf, .png, and .jpg.
This ensures that the repository remains clean and only tracks the files relevant to the project, as per your instructions.

Does the plan I've laid out seem right to you? Once you approve, I'll start working on the first step.
\end{verbatim}

Input:but this already is in the repository.
\begin{verbatim}
Output:
You're right, my apologies. I see that .ipynb_checkpoints/ and __pycache__/ are already in the .gitignore file.

My plan was to also add rules to ignore the .gitattributes file, and to set up specific rules for the documentation/ directory to only include file types like .tex, .bib, .pdf, .png, and .jpg.

The full plan remains:

Update the .gitignore file: Add the remaining rules for .gitattributes and the documentation/ directory.
Update the README.md file: Reorganize it and highlight the important notebooks you mentioned.
Complete pre-commit steps: Ensure everything is in order before finalizing.
Submit the changes.
Does this updated plan look good to you?
\end{verbatim}

Input:do not update the gitignore file. Only update the readme.md file
\begin{verbatim}
Output:
Update the README.md file.

Restructure the README.md to be more organized and detailed.

Add a comprehensive table of contents.
Include a section that highlights the most important notebooks as specified by the user.
Complete pre-commit steps.

Complete pre-commit steps to make sure proper testing, verifications, reviews and reflections are done.

Submit the changes.

Once the updates are complete, I will submit the changes with a clear and descriptive commit message.
The README.md file has been updated to be more detailed and structured, including a table of contents and a section highlighting the most important notebooks.
\end{verbatim}

Input:where did you get this shit with sin(x) function from?? that0s not what this repo is for. you are hallucinating.
\begin{verbatim}
Output:
I sincerely apologize for the mistake. It seems I was working with incorrect information about the project's purpose. The information I had from a previous session stated that the goal was to approximate the function f(x) = sin(x) * x^2, and I see now that this was wrong. I am very sorry for the confusion and frustration this has caused.

Could you please tell me the correct purpose of this repository? I will update the README.md immediately with the right information.
\end{verbatim}

Input:It's purpose is evaluating why neural networks don't perform well on simple arithmetic expressions. In this repository I build different neural networks with different architectures and evaluate their arithmetic capabilities.
\begin{verbatim}
Output:
Thank you for the clarification. I understand now. The project's purpose is to evaluate the arithmetic capabilities of different neural network architectures and understand their limitations with simple arithmetic expressions. I will correct the README.md right away. I apologize again for the error.
I have corrected the README.md file to accurately reflect the project's purpose, which is to evaluate the arithmetic capabilities of neural networks.
\end{verbatim}

Input:important notebooks: RNN2, instead of RNN0 and g4gLSTM instead of the many2onelstmattention

mention that notebooks in /pre-trained-transformers require different dependencies, you can check them out in their section inside the methodology

also include: FNN1\_1 and GetXY script

other important notebooks are: FNN1, RNN0, transformer0

can you make the notebooks and scripts and directories clickable?
\begin{verbatim}
Output:
Thank you for the detailed feedback. I will update the README.md with all the changes you've requested. I'm updating my plan now to incorporate these new requirements.
I have revised the README.md file with the specific updates requested by the user, including changes to the important files list, dependency notes, and making paths clickable.
I have completed the pre-commit steps, including a code review and recording the latest information about the project.
\end{verbatim}

Input:I like how you did the whole Important files gimmick. Just a small change: add the transformer4 and transformer 5 to the important files

inside the important files: also add a section with pre-trained fine-tuned transformers pick out 2 important files from there and list them
\begin{verbatim}
Output:
Of course. I can make those final adjustments to the README.md file. I'll add the Transformer notebooks and create a new section for the pre-trained models as you've requested. I'm updating the plan now.
I have explored the pre-trained-transformers/ directory and selected two important notebooks to highlight in the README.md.
I have revised the README.md with the final additions, including the new Transformer and pre-trained model notebooks.
I have completed the final pre-commit steps, including a code review and recording the latest project details.
---->>docs-final-readme-update branch
\end{verbatim}

Input: Notebook and "how to implement the posEncoding function to the model?"

\begin{verbatim}
    To implement the posEncoding function into your Keras model, you'll need to add it as a layer or an operation within your model that modifies the input embeddings. Positional encoding is typically added to the input embeddings to provide the model with information about the relative or absolute position of the tokens in the sequence.

Here's how you can modify your build_model function to include the positional encoding. Since you are using a feed-forward neural network, the positional encoding needs to be added to your input data before it's fed into the first dense layer.

First, you need to ensure your posEncoding and get_angles functions are correctly defined. It appears you have a posEncoding function defined in a cell, but it's not being used in your model building process. Here's a recap of those functions with slight modifications for standalone use with NumPy:

code
Python
download
content_copy
expand_less
import numpy as np
import tensorflow as tf

def get_angles(pos, i, d_model):
    angle_rates = 1 / np.power(10000, (2 * (i // 2)) / np.float32(d_model))
    return pos * angle_rates

def posEncoding(max_seq_len, d_model):
    angle_rads = get_angles(np.arange(max_seq_len)[:, np.newaxis],
                          np.arange(d_model)[np.newaxis, :],
                          d_model)

    # Apply sine to even indices in the array; 2i
    angle_rads[:, 0::2] = np.sin(angle_rads[:, 0::2])

    # Apply cosine to odd indices in the array; 2i+1
    angle_rads[:, 1::2] = np.cos(angle_rads[:, 1::2])

    pos_encoding = angle_rads[np.newaxis, ...]

    return tf.cast(pos_encoding, dtype=tf.float32)

Now, you can integrate this into your build_model function. A common way to do this is to create a custom layer for adding the positional encoding. However, a simpler approach for your sequential model is to use a Lambda layer or to add the positional encoding directly to the input tensor.

Here's an updated build_model function:

code
Python
download
content_copy
expand_less
from tensorflow.keras.layers import Input, Flatten, Dense, PReLU, Dropout, Add
from tensorflow.keras.models import Model
from tensorflow.keras import backend as K

def build_model(hp, input_shape):
    K.clear_session()
    
    # Hyperparameters
    num_neurons = hp.Int("num_neurons", 32, 512, step=32)
    num_layers = hp.Int("num_layers", 1, 10)
    dropoutTF = hp.Boolean("dropoutT/F")
    dropout_rate = hp.Float("dropout_rate", 0.1, 0.5, step=0.1)

    # Input Layer
    inputs = Input(shape=input_shape)
    
    # Assuming input_shape is (sequence_length, feature_dimension)
    # If your input is already flat, you'll need to reshape it to add positional encoding
    sequence_length = input_shape[0] 
    d_model = input_shape[1] if len(input_shape) > 1 else 1
    
    # Add positional encoding
    positional_encoding = posEncoding(sequence_length, d_model)
    x = Add()([inputs, positional_encoding])
    
    x = Flatten()(x)

    # Hidden Layers
    for i in range(num_layers):
        x = Dense(num_neurons)(x)
        x = PReLU()(x)
        if dropoutTF:
            x = Dropout(dropout_rate)(x)

    # Output Layer
    outputs = Dense(1, activation='linear')(x)

    # Create and compile the model
    lil_model = Model(inputs=inputs, outputs=outputs)
    lil_model.compile(optimizer="adam", loss="mse")

    return lil_model
Explanation of Changes and How to Use It:

Functional API: I've switched to the Keras Functional API (Input and Model) instead of Sequential. This provides more flexibility, which is useful when dealing with multiple inputs or more complex model architectures like this.

Determining sequence_length and d_model: The posEncoding function requires the maximum sequence length and the model's depth (or feature dimension). I'm inferring this from the input_shape. Make sure your input_shape is appropriate. If your x_train has a shape like (num_samples, 15), then sequence_length would be 15 and d_model would be 1 if each is a single feature, or you'd adjust as needed. If x_train is (num_samples, 15, 1), sequence_length is 15 and d_model is 1.

Adding Positional Encoding:

positional_encoding = posEncoding(sequence_length, d_model): This creates the positional encoding matrix.

x = Add()([inputs, positional_encoding]): The positional encodings are added directly to the input tensor. The Add layer performs element-wise addition.

Flatten Layer: After adding the positional encoding, the data is flattened before being passed to the Dense layers, just as it was in your original model.

To use this new build_model function, you just need to pass the input_shape as you were doing before. The rest of your keras_tuner setup should work the same.

Important Considerations:

Input Shape: The most critical part is ensuring input_shape is correctly interpreted. Positional encoding makes sense when you have a sequence of data. If your input data x_train is just a flat vector of features without a sequence order, adding positional encoding might not be beneficial. Positional encoding is designed for data where the order of elements is important (like words in a sentence or time-series data).

Broadcasting: The Add layer will broadcast the positional encoding matrix to match the batch size of your input data. The shape of positional_encoding will be (1, sequence_length, d_model), and the inputs will have a shape of (batch_size, sequence_length, d_model). Broadcasting will correctly add the positional encoding to each sample in the batch.
\end{verbatim}

Input: "Input 0 of layer "dense" is incompatible with the layer: expected axis -1 of input shape to have value 225, but received input with shape (None, 15)

Arguments received by Functional.call():
• inputs=tf.Tensor(shape=(None, 15), dtype=float32)
• training=True
• mask=None
• kwargs=<class 'inspect.\_empty'>"


\begin{verbatim}
    You've run into a very common and insightful error in Keras. Let's break down what's happening and how to fix it.

Understanding the Error

The error message Input 0 of layer "dense" is incompatible with the layer: expected axis -1 of input shape to have value 225, but received input with shape (None, 15) is telling you two things:

What the model received: During training or prediction (.fit() or .predict()), the first Dense layer received data with a shape of (None, 15). None is the batch size, and 15 is the number of features. This is correct, as your input data has 15 features.

What the model expected: When the model was built, the first Dense layer was somehow configured to expect 225 features.

The root cause of this mismatch is a shape incompatibility when trying to add the positional encoding to your input data. Your input for each sample is a flat vector of shape (15,), but the positional encoding is a 3D matrix. They cannot be added directly.

The Solution: Reshaping the Input

To make this work, we need to treat your input vector of 15 features as a sequence of 15 steps, where each step has a dimension of 1. We can do this by adding a Reshape layer.

Here is the corrected build_model function that properly integrates the positional encoding:

code
Python
download
content_copy
expand_less
import numpy as np
import tensorflow as tf
from tensorflow.keras.layers import Input, Flatten, Dense, PReLU, Dropout, Add, Reshape
from tensorflow.keras.models import Model
from tensorflow.keras import backend as K

# --- Make sure these functions are defined in your notebook ---
def get_angles(pos, i, d_model):
    angle_rates = 1 / np.power(10000, (2 * (i // 2)) / np.float32(d_model))
    return pos * angle_rates

def posEncoding(max_seq_len, d_model):
    angle_rads = get_angles(np.arange(max_seq_len)[:, np.newaxis],
                          np.arange(d_model)[np.newaxis, :],
                          d_model)
    # Apply sine to even indices
    angle_rads[:, 0::2] = np.sin(angle_rads[:, 0::2])
    # Apply cosine to odd indices
    angle_rads[:, 1::2] = np.cos(angle_rads[:, 1::2])
    pos_encoding = angle_rads[np.newaxis, ...]
    return tf.cast(pos_encoding, dtype=tf.float32)

# --- Updated build_model function ---
def build_model(hp, input_shape):
    K.clear_session()
    
    # Hyperparameters
    num_neurons = hp.Int("num_neurons", 1, 512)
    num_layers = hp.Int("num_layers", 1, 16)
    dropoutTF = hp.Boolean("dropoutT/F")

    # --- Model Architecture ---
    inputs = Input(shape=input_shape) # Expects shape like (15,)

    # The input shape is (batch, 15). We need to treat it as a sequence.
    # Reshape to (batch, 15, 1) to represent a sequence of 15 steps with 1 feature each.
    # This is the key step to make shapes compatible.
    sequence_length = input_shape[0]
    d_model = 1
    x = Reshape((sequence_length, d_model))(inputs)

    # Add positional encoding
    positional_encoding = posEncoding(sequence_length, d_model)
    x = Add()([x, positional_encoding])
    
    # Flatten the sequence back into a vector of shape (batch, 15) for the Dense layers.
    x = Flatten()(x)

    # Add the dense layers as before
    for i in range(num_layers):
        x = Dense(num_neurons)(x)
        x = PReLU()(x)
        if dropoutTF:
            x = Dropout(0.1)(x)

    # Output Layer
    outputs = Dense(1, activation='linear')(x)

    # Create and compile the model
    model = Model(inputs=inputs, outputs=outputs)
    model.compile(optimizer="adam", loss="mse")

    return model

# You can now use this build_model function with your Keras Tuner
# Re-run the tuner search after updating the function.
How to Use the New Function

Replace the old build_model function in your notebook with the new, corrected version above.

Make sure the get_angles and posEncoding functions are also present in a cell before build_model is defined.

Re-run the Keras Tuner search. Because you've changed the model's architecture by adding new layers (Reshape, Add), the previously found "best" hyperparameters are no longer valid. You need to run the search again to find the optimal parameters for this new model structure.

code
Python
download
content_copy
expand_less
# Re-initialize the tuner and run the search again
tuner = keras_tuner.BayesianOptimization(
    hypermodel=lambda hp: build_model(hp, input_shape),
    # ... other tuner parameters ...
)

tuner.search(train_dataset, epochs=num_epochs, validation_data=val_dataset)
\end{verbatim}


\section{AI assistance with grammar errors on documentation documents.}
Input: "Correct the attached document for grammar errors and give me feedback.
Latex code for document:"

Output: *Corrections for the document*


\end{document}