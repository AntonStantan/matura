\documentclass{article}

\title{ai_assistance}
\author{Anton Mukin}
\date{\today}

\setlength{\parindent}{0pt}


\begin{document}


AI was used lots throughout this project. Almost only for minor things 
e.g., correcting an error in the code, inspiration on how to code something 
suggestions to improve performance and so on.
As I think AI's don't perform well with big tasks and most of the time it's 
easier to do it yourself or find a better solution online.

LLMs used: MistralAI, Gemini2.5 flash, Gemini 2.5 pro, Qwen3, a little bit 
of copilot.
\\[2em]

Bigger tasks AI did: 
\\[2em]

Gemini2.5 Pro:

Input: "Write me a python function, which is a custom learning rate 
scheduler for a neural network. It should consist of a linear warmup and a 
cosine decay."

Output:
\begin{verbatim}
    

# Define a custom learning rate schedule class with warmup and cosine decay
class WarmupCosineDecay(tf.keras.optimizers.schedules.LearningRateSchedule):
    """
    A custom learning rate schedule that implements a linear warmup
    followed by a cosine decay.
    """
    def __init__(self, peak_lr, warmup_steps, decay_steps, alpha=0.0, name=None):
        super().__init__()
        self.peak_lr = peak_lr
        self.warmup_steps = warmup_steps
        self.decay_steps = decay_steps
        self.alpha = alpha
        self.name = name

    def __call__(self, step):
        with tf.name_scope(self.name or "WarmupCosineDecay"):
            # Ensure step is a float for calculations
            step = tf.cast(step, tf.float32)
            
            # --- 1. Warmup Phase ---
            # Linearly increase the learning rate from 0 to peak_lr
            warmup_lr = self.peak_lr * (step / self.warmup_steps)

            # --- 2. Cosine Decay Phase ---
            # Define the cosine decay schedule
            cosine_decay_schedule = tf.keras.optimizers.schedules.CosineDecay(
                initial_learning_rate=self.peak_lr,
                decay_steps=self.decay_steps,
                alpha=self.alpha
            )
            # Calculate the learning rate for the decay phase.
            # Note: The 'step' for the cosine part must be relative to its start.
            decay_lr = cosine_decay_schedule(step - self.warmup_steps)

            # --- 3. Choose the correct phase ---
            # Use tf.where to select the learning rate based on the current step
            learning_rate = tf.where(
                step < self.warmup_steps,
                warmup_lr,
                decay_lr
            )
            return learning_rate
\end{verbatim}

google jules:

Input: make a gitignore file for this repository. do not remove the direcotries "not\_relevant", they are relevant. Remove stuff like ipynbcheckpoints or pycashe
\begin{verbatim}
Output:
# Ignore Jupyter Notebook checkpoints
.ipynb_checkpoints/

# Ignore Python bytecode
__pycache__/
\end{verbatim}
Input: all the irrelevant files except for .tex, .bib, .pdf, .png, .jpg should be ignored too. also ignore gitattributes
\begin{verbatim}
Output: 
I've received the latest code review feedback, and I see a potential contradiction in the instructions for the .gitignore file.

The repository's main purpose seems to be machine learning, and we've identified several important Python (.py) and Jupyter Notebook (.ipynb) files. However, your latest instruction was to create a .gitignore that ignores all files except for .tex, .bib, .pdf, .png, and .jpg.

If I follow this instruction literally, all the Python and Notebook source files will be ignored by Git, which seems counterproductive to the project.

Could you please clarify? Should I:

Proceed with ignoring all files except .tex, .bib, .pdf, .png, and .jpg?
Also include .py and .ipynb files in the whitelist of files to keep?
Take a different approach to the .gitignore file?
Once I have this clarification, I can finalize the .gitignore file correctly.
\end{verbatim}
Input: I meant all the irrelevant files except .tex, .bib, .pdf, .png, and .jpg in the documentation directory.
\begin{verbatim}
Output: gitignore files were published in a branch "add-gitigrnore-rules"
\end{verbatim}
google jules:

Input:update the README.md file
\begin{verbatim}
Output:
Update the .gitignore file.

Add entries to ignore .ipynb_checkpoints/, __pycache__/, and .gitattributes.

Add rules to ignore all files in the documentation/ directory except for .tex, .bib, .pdf, .png, and .jpg files.
Update the README.md file.

Restructure the README.md to be more organized and detailed.

Add a comprehensive table of contents.
Include a section that highlights the most important notebooks as specified by the user.
Complete pre-commit steps.

Complete pre-commit steps to make sure proper testing, verifications, reviews and reflections are done.

Submit the changes.

Once the updates are complete, I will submit the changes with a clear and descriptive commit message.
\end{verbatim}

Input:why update the gitignore?
\begin{verbatim}
Output:
I'm updating the .gitignore file to incorporate the preferences you shared earlier. Specifically, I'll be adding rules to:

Ignore the .ipynb_checkpoints/, __pycache__/, and .gitattributes files.
In the documentation/ directory, ignore all files except for specific types like .tex, .bib, .pdf, .png, and .jpg.
This ensures that the repository remains clean and only tracks the files relevant to the project, as per your instructions.

Does the plan I've laid out seem right to you? Once you approve, I'll start working on the first step.
\end{verbatim}

Input:but this already is in the repository.
\begin{verbatim}
Output:
You're right, my apologies. I see that .ipynb_checkpoints/ and __pycache__/ are already in the .gitignore file.

My plan was to also add rules to ignore the .gitattributes file, and to set up specific rules for the documentation/ directory to only include file types like .tex, .bib, .pdf, .png, and .jpg.

The full plan remains:

Update the .gitignore file: Add the remaining rules for .gitattributes and the documentation/ directory.
Update the README.md file: Reorganize it and highlight the important notebooks you mentioned.
Complete pre-commit steps: Ensure everything is in order before finalizing.
Submit the changes.
Does this updated plan look good to you?
\end{verbatim}

Input:do not update the gitignore file. Only update the readme.md file
\begin{verbatim}
Output:
Update the README.md file.

Restructure the README.md to be more organized and detailed.

Add a comprehensive table of contents.
Include a section that highlights the most important notebooks as specified by the user.
Complete pre-commit steps.

Complete pre-commit steps to make sure proper testing, verifications, reviews and reflections are done.

Submit the changes.

Once the updates are complete, I will submit the changes with a clear and descriptive commit message.
The README.md file has been updated to be more detailed and structured, including a table of contents and a section highlighting the most important notebooks.
\end{verbatim}

Input:where did you get this shit with sin(x) function from?? that0s not what this repo is for. you are hallucinating.
\begin{verbatim}
Output:
I sincerely apologize for the mistake. It seems I was working with incorrect information about the project's purpose. The information I had from a previous session stated that the goal was to approximate the function f(x) = sin(x) * x^2, and I see now that this was wrong. I am very sorry for the confusion and frustration this has caused.

Could you please tell me the correct purpose of this repository? I will update the README.md immediately with the right information.
\end{verbatim}

Input:It's purpose is evaluating why neural networks don't perform well on simple arithmetic expressions. In this repository I build different neural networks with different architectures and evaluate their arithmetic capabilities.
\begin{verbatim}
Output:
Thank you for the clarification. I understand now. The project's purpose is to evaluate the arithmetic capabilities of different neural network architectures and understand their limitations with simple arithmetic expressions. I will correct the README.md right away. I apologize again for the error.
I have corrected the README.md file to accurately reflect the project's purpose, which is to evaluate the arithmetic capabilities of neural networks.
\end{verbatim}

Input:important notebooks: RNN2, instead of RNN0 and g4gLSTM instead of the many2onelstmattention

mention that notebooks in /pre-trained-transformers require different dependencies, you can check them out in their section inside the methodology

also include: FNN1\_1 and GetXY script

other important notebooks are: FNN1, RNN0, transformer0

can you make the notebooks and scripts and directories clickable?
\begin{verbatim}
Output:
Thank you for the detailed feedback. I will update the README.md with all the changes you've requested. I'm updating my plan now to incorporate these new requirements.
I have revised the README.md file with the specific updates requested by the user, including changes to the important files list, dependency notes, and making paths clickable.
I have completed the pre-commit steps, including a code review and recording the latest information about the project.
\end{verbatim}

Input:I like how you did the whole Important files gimmick. Just a small change: add the transformer4 and transformer 5 to the important files

inside the important files: also add a section with pre-trained fine-tuned transformers pick out 2 important files from there and list them
\begin{verbatim}
Output:
Of course. I can make those final adjustments to the README.md file. I'll add the Transformer notebooks and create a new section for the pre-trained models as you've requested. I'm updating the plan now.
I have explored the pre-trained-transformers/ directory and selected two important notebooks to highlight in the README.md.
I have revised the README.md with the final additions, including the new Transformer and pre-trained model notebooks.
I have completed the final pre-commit steps, including a code review and recording the latest project details.
---->>docs-final-readme-update branch
\end{verbatim}


\end{document}